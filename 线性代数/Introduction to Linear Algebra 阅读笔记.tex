\documentclass{ctexart}
\usepackage{ctex}
\usepackage{amsmath, amsthm, amssymb, amsfonts}
\usepackage{thmtools}
\usepackage{graphicx}
\usepackage{setspace}
\usepackage{geometry}
\usepackage{float}
\usepackage[colorlinks=true]{hyperref}
\usepackage[utf8]{inputenc}
\usepackage[english]{babel}
\usepackage{framed}
\usepackage[dvipsnames]{xcolor}
\usepackage{tcolorbox}

\colorlet{LightGray}{White!90!Periwinkle}
\colorlet{LightOrange}{Orange!15}
\colorlet{LightGreen}{Green!15}

\newcommand{\HRule}[1]{\rule{\linewidth}{#1}}

\declaretheoremstyle[name=Theorem,]{thmsty}
\declaretheorem[style=thmsty,numberwithin=section]{theorem}
\tcolorboxenvironment{theorem}{colback=LightGray}

\declaretheoremstyle[name=Proposition,]{prosty}
\declaretheorem[style=prosty,numberlike=theorem]{proposition}
\tcolorboxenvironment{proposition}{colback=LightOrange}

\declaretheoremstyle[name=Principle,]{prcpsty}
\declaretheorem[style=prcpsty,numberlike=theorem]{principle}
\tcolorboxenvironment{principle}{colback=LightGreen}

\setstretch{1.2}
\geometry{
  paper=a4paper,
  includeheadfoot,
  textheight=9in,
  textwidth=5.5in,
  top=1in,
  headheight=12pt,
  headsep=25pt,
  footskip=30pt
}
\usepackage{indentfirst}
\setlength{\parindent}{2.45em}
\setcounter{tocdepth}{3}
% ------------------------------------------------------------------------------

\begin{document}

% ------------------------------------------------------------------------------
% Cover Page and ToC
% ------------------------------------------------------------------------------

\title{ \normalsize \textsc{}
\\ [2.0cm]
\HRule{1.5pt} \\
\LARGE \textbf{\uppercase{Introduction to Linear Algebra 笔记}
\HRule{2.0pt} \\ [0.6cm] \vspace*{10\baselineskip}}
}
\date{}
\author{\textbf{Author} \\
  Wen Gao \\
  Wuhan \\
  \today}

\maketitle
\newpage

\tableofcontents
\newpage

% ------------------------------------------------------------------------------

% \section{Examples}

% \begin{theorem}
%   This is a theorem.
% \end{theorem}

% \begin{proposition}
%   This is a proposition.
% \end{proposition}

% \begin{principle}
%   This is a principle.
% \end{principle}

% % Maybe I need to add one more part: Examples.
% % Set style and colour later.

% \subsection{Pictures}

% \begin{figure}[htbp]
%   \center
%   \includegraphics[scale=0.06]{img/photo.jpg}
%   \caption{Sydney, NSW}
% \end{figure}

% \subsection{Citation}

% This is a citation\cite{Eg}.

% \newpage
\section{向量简介}
\subsection{向量和线性组合}
\subsection{向量长度和向量点乘}
\subsection{矩阵}
\section{求解线性方程组}
\section{向量空间和子空间}
\section{正交性}
\section{行列式}
\section{特征值和特征向量}
\section{奇异值分解(SVD)}
\section{线性变换}
\section{复矩阵}
\section{线性代数的应用}
\section{数值线性代数}
\section{概率统计中的线性代数}
% ------------------------------------------------------------------------------
% Reference and Cited Works
% ------------------------------------------------------------------------------

% \bibliographystyle{IEEEtran}
% \bibliography{References.bib}

% ------------------------------------------------------------------------------
\end{document}
